\documentclass{beamer}
\usepackage{advDSpreamble}

\setbeamertemplate{footline}[frame number]
\setbeamertemplate{headline}{
  \leavevmode%
  \hbox{%
    \begin{beamercolorbox}[wd=.5\paperwidth,ht=2.25ex,dp=1ex,left]{section in head/foot}%
      \usebeamerfont{section in head/foot}\insertsectionhead%
    \end{beamercolorbox}%
    \begin{beamercolorbox}[wd=.5\paperwidth,ht=2.25ex,dp=1ex,right]{subsection in head/foot}%
      \usebeamerfont{subsection in head/foot}\insertsubsectionhead%
    \end{beamercolorbox}%
  }%
  \vskip0pt%
}

\usepackage[style=ieee]{biblatex}
\setbeamertemplate{bibliography item}{\insertbiblabel}

\usepackage{filecontents}
\addbibresource{\jobname.bib}

\everymath{\displaystyle}
% Use the Warsaw theme
\usetheme{Warsaw}
\setbeamertemplate{theorems}[numbered]

\newcommand{\EDCS}{\text{EDCS}}

\title{Maximal Matching Problems Using EDCS}
\author{Team 7}
\date{26th, May, 2025}
\usefonttheme{professionalfonts}
\begin{document}

\begin{frame}
  \titlepage
\end{frame}

\begin{frame}
  \frametitle{Table of Contents}
  \tableofcontents
\end{frame}

\section{Introduction}
\begin{frame}{Definition of EDCS}
    We first give some basic definitions.
    \begin{definition}[\cite{BS15}]
        For any graph $G(V,E)$ and integers $\beta\geq\beta^-\geq0$, 
        an edge-degree constrained subgraph (EDCS) $(G,\beta,\beta^-)$ is a subgraph 
        $H:=(V,E(H))$ of $G$ with the following two properties:
        \begin{enumerate}[label=(P\arabic*)]
            \item For any edge $(u,v)\in E(H)$: $\deg_H(u)+\deg_H(v)\leq\beta$.
            \item For any edge $(u,v)\in E\exc E(H)$: $\deg_H(u)+\deg_H(v)\geq\beta^-$.
        \end{enumerate}
    \end{definition}
\end{frame}

\begin{frame}{Existence of EDCS}
    \begin{theorem}
        This is a theorem of EDCS.
    \end{theorem}
\end{frame}

\section{Maximal Matching Problems}
\begin{frame}{EDCS on Dynamic Graphs}
    EDCS is good on bipartites.
    \begin{lemma}
        Given a bipartite $G$, $H=\EDCS$ achieves $\mu(G)\leq\left( \frac{3}{2}+\varepsilon \right)\mu(H)$.
    \end{lemma}   
    \begin{proof}
        We try to find the correct constant.
    \end{proof} 
\end{frame}

\begin{frame}{EDCS on Dynamic Graphs}
    EDCS is good on general graphs.
    \begin{lemma}[\cite{AB19}]
        Let $G(V, E)$ be any graph and $\varepsilon < 1/2$ be a parameter. 
        For $\lambda \leq \frac{\varepsilon}{32}$, $\beta \geq 8 \lambda^{-2} \log(1/\lambda)$, and $\beta^{-} \geq (1 - \lambda) \cdot \beta$, 
        in any subgraph $H := \mathrm{EDCS}(G, \beta, \beta^-)$, $$\mu(G) \leq \left(\frac{3}{2} + \varepsilon\right) \cdot \mu(H).$$
    \end{lemma}
\end{frame}

\begin{frame}{One-way Communication Protocal}
    \begin{definition}
        One-way communication protocol is a protocol where the sender sends a message to the receiver, 
        and the receiver does something.
    \end{definition}
    This is related to stochastic graphs.
\end{frame}

\begin{frame}{EDCS on Stochastic Graphs}
    EDCS is good on stochastic graphs.
    \begin{lemma}
        Given a stochastic graph $G$, $H=\EDCS$ achieves $\mu(G)\leq\left( \frac{3}{2}+\varepsilon \right)\mu(H)$.
    \end{lemma}
\end{frame}

\begin{frame}{EDCS on Fault-tolerent Graphs}
    EDCS is good on Fault-tolerent graphs.
    \begin{lemma}
        Given a Fault-tolerent graph $G$, $H=\EDCS$ achieves $\mu(G)\leq\left( \frac{3}{2}+\varepsilon \right)\mu(H)$.
    \end{lemma}
\end{frame}

\section{Promising Directions}
\begin{frame}{Some Ideas}
    EDCS is a greedy algorithm. Maybe we can be less greedy.
    \begin{block}{Idea}
        We can try to find a better algorithm.        
    \end{block}
\end{frame} 

\begin{frame}{bibliography}
    \printbibliography
\end{frame}

\end{document}